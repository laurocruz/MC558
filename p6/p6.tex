\documentclass[12pt,a4paper]{article}
\usepackage[utf8]{inputenc}
\usepackage{amsmath}
\usepackage{breqn}
\usepackage{amsfonts}
\usepackage{amssymb}
\usepackage{graphicx}
\usepackage[margin=0.1in]{geometry}

\title{MC558 - P6}
\author{Lauro Cruz e Souza - 156175 }
\date{June 2017}

\begin{document}

\maketitle

\section{Variáveis:}

Para $ i \in \{1,...,13\} $:

$ x[i] $ = Rendimento no mês $i$

$ CC[i] $ = Conta corrente no mês $i$

$ P[i] $  = Poupança no mês $i$

$ CDB[i] $ = CDB no mês $i$

$ AC[i] $ = Dinheiro investido em acões no mês $i$\\

Para $ i \in \{1,...,12\} $:

$ retP[i] = $ Quantidade de dinheiro a ser retirado da poupança no mês $i+1$

$ retCDB[i] = $ Quantidade de dinheiro a ser retirado do CDB no mês $i+1$

$ retAC[i] = $ Quantidade de dinheiro a ser retirado das ações no mês $i+1$

$ invP[i] = $ Quantidade de dinheiro a ser investido na poupança no mês $i$

$ invCDB[i] = $ Quantidade de dinheiro a ser investido no CDB no mês $i$

$ invAC[i] = $ Quantidade de dinheiro a ser investido em ações no mês $i$ \\\\

\section{Maximização:}
$$ h + \sum_{i=1}^{13} x[i] $$

O total de dinheiro inicialmente (herança $h$) mais o rendimento de cada um dos meses do ano (e janeiro de 2019) é o total de dinheiro final, ou seja, queremos maximizar o total de dinheiro ao final de um ano.\\

\section{Restrições:}

\textbf{O rendimento inicial é sempre negativo:}

$ x[0] = -d[0] $

\textbf{Os proximos rendimentos são calculados pela quantidade de dinheiro atual menos a do mês anterior:}

$ x[i] = CC[i] + P[i] + CDB[i] + AC[i] - CC[i-1] - P[i-1] - CDB[i-1] - AC[i-1]$ $\forall i \in \{2,..,13\} $ \\


\textbf{O rendimento precisa ser maior que o negativo da quantidade de dinheiro disponível para não termos saldo negativo:}

$ x[0] \geq -h $

$ x[i] \geq -(CC[i-1] + P[i-1] + CDB[i-1] + AC[i-1]) $ $\forall i \in \{2,...,13\}$\\


\textbf{A conta corrente inicial é a herança menos os investimentos iniciais:}

$ CC[0] = h - invP[0] - invCDB[0] - invAC[0] - d[0] $

\textbf{Para o resto, a conta corrente é o saldo do mês anterior mais a quantidade retirada dos investimentos, menos a quantidade investida e as tarifas:}

$ CC[i] = CC[i-1] + retP[i-1] + retCDB[i-1] + retAC[i-1] - invP[i] - invCDB[i] - invAC[i] - c - d[i]$ $\forall i \in \{2,..,13\}$\\


\textbf{O saldo inicial da poupança é a quantidade investida:}

$ P[0] = invP[0] $

\textbf{O resto é o saldo do mês anterior com a quantidade de rendimento, mais o novo montante investido menos a quantidade retirada e sua taxa:}

$ P[i] = ((rp/100)+1)*P[i-1] + invP[i] - ((tp/100)+1)*retP[i-1])$ $\forall i \in \{2,..,13\}$\\


\textbf{O saldo inicial do CDB é a quantidade investida:}

$ CDB[0] = invCDB[0] $

\textbf{O resto é o saldo do mês anterior com a quantidade de rendimento, mais o novo montante investido menos a quantidade retirada e sua taxa:}

$ CDB[i] = ((rb[i-1]/100)+1)*CDB[i-1] + invCDB[i] - ((tb/100)+1)*retCDB[i-1]$ $\forall i \in \{2,..,13\}$\\


\textbf{O saldo inicial em ações é a quantidade investida:}

$ AC[0] = invAC[0] $

\textbf{O resto é o saldo do mês anterior com a quantidade de rendimento, mais o novo montante investido menos a quantidade retirada:}

$ AC[i] = ((ra[i-1]/100)+1)*AC[i-1] + invAC[i] - retAC[i-1]$ $\forall i \in \{2,..,13\}$\\


\textbf{A quantidade retirada dos fundos de investimento no mês $i+1$ (mais a taxa de retiro no caso da poupança e do CDB) têm que ser menores ou iguais ao saldo atual do mês $i$ mais o rendimento equivalente a este mês:}

$ retP[i]*((tp/100)+1) \leq ((rp/100)+1)*P[i]$ $\forall i \in \{1,..,12\}$

$ retCDB[i]*((tb/100)+1) \leq ((rb[i]/100)+1)*CDB[i]$ $\forall i \in \{1,..,12\}$

$ retAC[i] \leq ((ra[i]/100)+1)*AC[i]$ $\forall i \in \{1,..,12\}$\\


\textbf{A quantidade investida no primeiro mês tem que ser menor ou igual que o dinheiro disponível (toda a herança menos a taxa do mês):}

$ invP[0] + invCDB[0] + invAC[0] \leq h - d[0] $

\textbf{Nos outros meses a quantidade investida tem que ser menor ou igual ao saldo da conta corrente mais a quantidade retirada dos fundos de investimento e menos as taxas:}

$ invP[i] + invCDB[i] + invAC[i] \leq CC[i-1] + retP[i-1] + retCDB[i-1] + retAC[i-1] - c - d[i]$ $\forall i \in \{2,..,12\}$\\


\textbf{Aqui temos os limites inferiores de todas as variáveis menos x, e todas obviamente tem limite inferior $0$:}

$ CC[i],P[i],CDB[i],AC[i] \geq 0$ $\forall i \in \{1,...,13\} $

$ invP[i],invCDB[i],invAC[i],retP[i],retCDB[i],retAC[i] \geq 0$ $\forall i \in \{1,...,12\} $\\\\


\section{Resultados:}

Os resultados do laboratório apresentam os valores ótimos esperados, mas a resolução (quantidades investidas e retiradas ao longo dos meses) diferem dos testes em alguns casos.\\\\
Nos testes 7 e 8 houveram problemas pequenos de aproximação em dois casos ($\approx 0.02$ dinheiros).

\end{document}